\makenoidxglossaries

\newacronym{elen}{ELEN}{Maximum Element Length. Defines the biggest size (in bits) any vector element can have.}

\newacronym{vlen}{VLEN}{Vector Register Length. Defines the biggest size (in bits) of any vector register.}

\newacronym{vrg}{VRG}{Vector Register Group. Any set of vector registers taken as a unity, even if such a set only contains one register.}

\newacronym{sew}{SEW}{Selected Element Length. Represents the global size (in bits) of the elements within any \acrshort{vrg}.}

\newacronym{lmul}{LMUL}{Vector Length Multiplier. Global indicator of how many vector registers are grouped together for any \acrshort{vrg}.}

\newacronym{eew}{EEW}{Effective Element Length. Size (in bits) of the elements within a particular \acrshort{vrg} operand when dealing with mixed-width or widening-narrowing operations.}

\newacronym{emul}{EMUL}{Effective Vector Length Multiplier. Indicates how many vector registers are grouped together for a particular \acrshort{vrg} operand when dealing with mixed-width or widening-narrowing operations}

\newacronym{avl}{AVL}{Application Vector Length. Indicates how many vector elements are to be processed in total for a given application.}

\newacronym{vlmax}{VLMAX}{Maximum Vector Length. Indicates the maximum number of elements that can be processed with our implementation's \acrshort{vlen} and the current \acrshort{sew} and \acrshort{lmul} settings.}

\newacronym{isa}{ISA}{Instruction Set Architecture. Defines the minimum hardware and software requirements a given implementation needs to cover in order to execute a set of instructions (?).}

\newacronym{csr}{CSR}{Control and Status Register. A special kind of register within a given memory-space that holds important information regarding the architecture's characteristics and its state (?).}